\documentclass{article}
\usepackage{lipsum}
\usepackage[margin=1in,left=1.5in,includefoot]{geometry}
\usepackage[hidelinks]{hyperref}

%Graphics
\usepackage{graphicx}
\usepackage{float}

%Header and footer Stuff
\usepackage{fancyhdr}
\pagestyle{fancy}
\fancyhead{}
\fancyfoot{}
\fancyfoot[R]{\thepage}
\renewcommand{\headrulewidth}{0pt}
\renewcommand{\footrulewidth}{0pt}

%
\begin{document}

\begin{titlepage}
    \begin{center}
        \line(1,0){300}\\
        [2mm]
        \huge{\bfseries Tobias the Groundhog}\\
        [2mm]
        \line(1,0){300}\\
        [1.5cm]
        \textsc{\LARGE A theZanShow Production}\\
        [0.75cm]
        \textsc{\Large On using LaTex to write a simple report}\\
        [10cm]
    \end{center}
    \begin{flushright}
        \textsc{\large Zan B.\\
        Latex User\\
        \#123456879\\
        June 26, 2014 \\}
    \end{flushright}
\end{titlepage}
%Summary

\pagenumbering{roman}
\section*{Summary}
\addcontentsline{toc}{section}{\numberline{}Summary}
This is the summary section. This teaches you how to write a report in LaTeX.
\cleardoublepage

\section*{Acknowledgements}
\addcontentsline{toc}{section}{\numberline{}Acknowledgements}
Thanks to the LaTeX creators.
\cleardoublepage

%Table of contents
\tableofcontents
\thispagestyle{empty}
\cleardoublepage
%end Table of contents

%List of figures
\listoffigures
\addcontentsline{toc}{section}{\numberline{}List of figures}
\cleardoublepage

\listoftables
\addcontentsline{toc}{section}{\numberline{}List of tables}
\cleardoublepage

%main body
\pagenumbering{arabic}
\setcounter{page}{1}

\section{Introduction}\label{sec:intro}
This is the first line of the report. This report will document on Tobias, the Groundhog, who lives in my back yard.

Tobias likes to live in a hole in the ground. This text will wrap around properly, don't you worry.
\lipsum[1]

\newpage
\section{Tobias' Lifestyle}
Tobias, although he lives in dole in the ground, he also likes to climb trees.

The introduction is found on page \pageref{sec:intro}

\begin{figure}[H]
    \centering
    \includegraphics [height=3in]{/home/professoroptics/Downloads/groundhog-850x560.jpg}
    \caption[Optional Caption]{Real, local caption}
    \label{fig:tobias}
\end{figure}
Figure \ref{fig:tobias} shows tobias in a tree.
\subsection{Where Tobias gets food}
Usually groundhogs nibble on greens found on the ground, but not Tobias. He is the exception to the rule.

\begin{table}[H]
    \centering
    \label{tab:tobiastreesightings}
    \caption[This is optional caption, without reference]{Local caption, with reference}
    \begin{tabular}{l c r}
        Date & In Tree? & Raining? \\ \hline
        April 26 & Yes & Yes \\
        June 7 & yes & no \\
        June 20 & yes & no \\
    \end{tabular}

\end{table}
Table \ref{tab:tobiastreesightings}
\subsubsection{Subsubsection}
This is a subsection.
\end{document}
